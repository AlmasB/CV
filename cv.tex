%%%%%%%%%%%%%%%%%%%%%%%%%%%%%%%%%%%%%%%%%
% Plasmati Graduate CV
% LaTeX Template
% Version 1.0 (24/3/13)
%
% This template has been downloaded from:
% http://www.LaTeXTemplates.com
%
% Original author:
% Alessandro Plasmati (alessandro.plasmati@gmail.com)
%
% CV author:
% Almas Baimagambetov (almaslvl@gmail.com)
%
% License:
% CC BY-NC-SA 3.0 (http://creativecommons.org/licenses/by-nc-sa/3.0/)
%
% Important note:
% This template needs to be compiled with XeLaTeX.
% The main document font is called Fontin and can be downloaded for free
% from here: http://www.exljbris.com/fontin.html
%
%%%%%%%%%%%%%%%%%%%%%%%%%%%%%%%%%%%%%%%%%

%----------------------------------------------------------------------------------------
%	PACKAGES AND OTHER DOCUMENT CONFIGURATIONS
%----------------------------------------------------------------------------------------

\documentclass[a4paper,11pt]{article} % Default font size and paper size

\usepackage[a4paper,bindingoffset=0.2in,%
            left=1in,right=1in,top=1in,bottom=1in,%
            footskip=.25in]{geometry}

\usepackage{fontspec} % For loading fonts
\defaultfontfeatures{Mapping=tex-text}
%\setmainfont{Fontin-SmallCaps.otf} % Main document font

\usepackage{xunicode,xltxtra,url,parskip} % Formatting packages

\usepackage[usenames,dvipsnames]{xcolor} % Required for specifying custom colors

%\usepackage[big]{layaureo} % Margin formatting of the A4 page, an alternative to layaureo can be \usepackage{fullpage}
% To reduce the height of the top margin uncomment: \addtolength{\voffset}{-1.3cm}

\usepackage{hyperref} % Required for adding links	and customizing them
\definecolor{linkcolour}{rgb}{0.1,0.3,1.0} % Link color
\hypersetup{colorlinks,breaklinks,urlcolor=linkcolour,linkcolor=linkcolour} % Set link colors throughout the document

\usepackage{titlesec} % Used to customize the \section command
\titleformat{\section}{\Large\scshape\raggedright}{}{0em}{}[\titlerule] % Text formatting of sections
\titlespacing{\section}{0pt}{3pt}{3pt} % Spacing around sections

\begin{document}

\pagestyle{empty} % Removes page numbering

\font\fb=''[cmr10]'' % Change the font of the \LaTeX command under the skills section

%----------------------------------------------------------------------------------------
%	NAME AND CONTACT INFORMATION
%----------------------------------------------------------------------------------------

\par{\centering{\Huge Almas Baimagambetov}\bigskip\par} % Your name

\section{Statement}

I am a Senior Lecturer and \textit{Computing and Games} Subject Co-Lead at the University of Brighton, where I teach on software and game development courses.
My PhD is in computer science and I remain an active researcher.
I am the author of \href{https://github.com/AlmasB/FXGL}{FXGL}, a game engine used by academic institutions to teach game development.
I lead and contribute to numerous projects on \href{https://github.com/AlmasB}{GitHub} and run an educational \href{https://www.youtube.com/almasb0/videos}{YouTube} channel for software and games.

%----------------------------------------------------------------------------------------
%	WORK EXPERIENCE 
%----------------------------------------------------------------------------------------

\section{Work Experience}

\begin{tabular}{r|p{11cm}}

Feb 2020- Present & Senior Lecturer in Computing, \emph{University of Brighton, UK}\\
\multicolumn{2}{c}{} \\

%------------------------------------------------

Sept 2018- Jan 2020 & Lecturer in Computing, \emph{University of Brighton, UK}\\
\multicolumn{2}{c}{} \\

%------------------------------------------------

Oct 2015- Aug 2018 & Part-time Lecturer in Computing, \emph{University of Brighton, UK}\\
\multicolumn{2}{c}{} \\

%------------------------------------------------

Oct 2012- Feb 2015 & Guest Speaker and Mentor (Volunteer), \emph{Bellerbys College, UK}\\
\multicolumn{2}{c}{} \\

July 2007- Oct 2015 & Software Developer (Freelance)\\
\multicolumn{2}{c}{} \\

\end{tabular}

%----------------------------------------------------------------------------------------
%	EDUCATION
%----------------------------------------------------------------------------------------

\section{Education}

\begin{tabular}{r|p{11cm}}

Feb 2019- Oct 2020 & \textbf{PGCert in L\&T HE}, \emph{University of Brighton, UK}\\
\multicolumn{2}{c}{} \\

%------------------------------------------------

July 2015- Dec 2019 & \textbf{PhD in Computer Science}, \emph{University of Brighton, UK}\\
& Thesis: automated visualization of grouped networks\\
& using Euler diagrams and graphs\\
& keywords: set theory, graph theory, topology,\\
& computational geometry \& graphics \\
\multicolumn{2}{c}{} \\

%------------------------------------------------
	
Oct 2012- July 2015 & \textbf{BSc Computer Science (Games)}, \emph{University of Brighton, UK}\\
& Analysis of software development issues in large scale games\\
&\normalsize Project grade: \textbf{A+} Degree: \textbf{1st Class Honours} \\
\multicolumn{2}{c}{} \\

\end{tabular}





\section{Projects}

Explore \href{https://github.com/AlmasB}{GitHub} for examples of projects using a range of programming languages.

The \href{https://github.com/AlmasB/FXGLGames}{FXGLGames} project is a collection of games developed using the FXGL framework in Java and Kotlin. The repository features a range of classic video games, including Space Invaders, Pac-man, Breakout and many more. All these games are open-source and suitable for both beginners and more experienced developers.

The \href{https://github.com/AlmasB/FXTutorials}{FXTutorials} project contains most of the JavaFX source code featured on the YouTube channel above. Having access to the source code of a video tutorial is beneficial for those who prefer to skim through the content, rather than follow alongside the tutorial.

\href{https://github.com/AlmasB/Zephyria}{Zephyria} is an RPG game written in Kotlin that uses the FXGL framework. This game is a sophisticated example that showcases many of FXGL's features, combined with heavy use of Kotlin DSL.

I also help co-maintain community-oriented (and community-driven) JavaFX projects, such as \href{https://github.com/gluonhq/scenebuilder}{Scene Builder}, \href{https://github.com/FXyz/FXyz}{FXyz}, \href{https://github.com/mhrimaz/AwesomeJavaFX}{AwesomeJavaFX}, and \href{https://github.com/FXDocs/docs}{FXDocs}.




\section{Awards}

\begin{tabular}{rl}

Jan 2022 & Robotics AI Lab Research Bid (£158 000), UoB\\

Dec 2021 & Belong at Brighton Events Support (£375), UoB\\

Oct 2019 & CLT Scholarship (£1 000), Centre of Learning \& Teaching, UoB\\

June 2018 & Best Student Paper, Diagrams 2018 Conference\\

Sept 2015 & International Research Scholarship (50\% fee reduction), UoB\\

July 2015 & Best Final Year Development Project (£250), The FDM Group\\

Nov 2014 & Academic Merit Based Scholarship (£1 000), UoB \\

Nov 2013 & Academic Merit Based Scholarship (£1 000), UoB \\

\end{tabular}

% Papers

\section{Research and Professional Talks (prev. 4 years)}

\begin{center}
\begin{tabular}{rl}

February 2022 & AI Pathfinding in FXGL\\ & (FOSDEM 2022 international software development conference) \\

July 2021 & FXGL: Cross-platform JavaFX Game Engine\\ & (JetBrains International Live) \\

April 2021 & High-performance Game Engine for Java and Kotlin \\ & (New York Java Group) \\

March 2021 & FXGL Game Engine\\ & (Silicon Valley JavaFX Group) \\

February 2021 & A Practical Introduction to FXGL\\ & (FOSDEM 2021 international software development conference) \\

January 2021 & A Practical Introduction to FXGL\\ & (Brighton Java Meetup) \\

November 2020 & Modern JavaFX Game Development with FXGL\\ & (JFX-Days international conference) \\

August 2020 & Evaluating Visualizations of Sets and Networks\\ & (11th International Conference on the Theory and Application of Diagrams) \\

July 2020 & Impact of Gamified Work-based Learning on Student Experience\\ & (Education and Student Experience Conference)\\

June 2019 & Automated Visualization of Grouped Networks Using Euler Diagrams and Graphs\\ & (CEM Conference) \\

April 2019 & Java and JavaFX Game Development\\ & (Brighton Java Meetup) \\

June 2018 & Generating Effective Euler Diagrams\\ & (10th International Conference on the Theory and Application of Diagrams) \\




% May 2017 & Novel Algorithm for Euler Diagram Generation\\ & (University of Brighton Internal Conference) \\

% March 2017 & Data Visualization Workshop\\ & (Presenter at Data Visualization Brighton Meetup) \\

% Feb. 2017 & An Inductive Approach to P-preserving Euler Diagram Generation\\ & (Visual Modelling Group Talk) \\

% June 2016 & Grouped Networks and Associated Challenges\\ & (University of Brighton Internal Conference) \\

% May 2016 & Euler Diagram Generation Techniques\\ & (Visual Modelling Group Talk) \\

\end{tabular}
\end{center}



\section{Publications}

1. \textbf{Baimagambetov, A.}, Stapleton, G., Blake, A. and Howse, J. (2020)
Evaluating Visualizations of Sets and Networks that Use Euler Diagrams and Graphs In:
11th International Conference on the Theory and Application of Diagrams, Tallinn, 24-28 August 2020.

2. \textbf{Baimagambetov, A.}, Howse, J., Stapleton, G. and Delaney, A. (2018)
Generating Effective Euler Diagrams In:
10th International Conference on the Theory and Application of Diagrams, Edinburgh, 18-22 June 2018.

3. \textbf{Baimagambetov, A.} (2018)
Automated Visualization of Grouped Networks In:
10th International Conference on the Theory and Application of Diagrams, Edinburgh, 18-22 June 2018.
(Graduate Symposium report).

\end{document}
